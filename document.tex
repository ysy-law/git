git参考文档

关联github仓库
第1步:创建SSH Key。在用户主目录下,看看有没有.ssh目录,
       如果有,再看看这个目录下有没有id_rsa和id_rsa.pub这两个文件,如果已经有了,可直接跳到下一步。
       如果没有,打开Shell(Windows下打开Git Bash),创建SSH Key:
       ssh-keygen -t rsa -C "email@demo.com" email地址	 
       你需要把邮件地址换成你自己的邮件地址,然后一路回车,使用默认值即可,由于这个Key也不是用于军事目的,所以也无需设置密码。
       如果一切顺利的话,可以在用户主目录里找到.ssh目录,里面有id_rsa和id_rsa.pub两个文件,
	   这两个就是SSH Key的秘钥对,id_rsa是私钥,不能泄露出去,id_rsa.pub是公钥,可以放心地告诉任何人。	
第2步:登陆GitHub,打开“Account settings”,“SSH Keys”页面:然后,点“Add SSH Key”,填上任意Title,在Key文本框里粘贴id_rsa.pub文件的内容	   

git init                   把当前目录变成Git可以管理的仓库
git status                 展示仓库当前的状态
git diff [HEAD --] demo.txt查看difference,显示格式是Unix通用的diff格式,加上参数对比上一次的
git add  demo.txt          提交到暂存区
git commit -m "captions"   提交到分支
git log [--pretty=oneline] 提交历史记录,加上参数会精简显示
git reset --hard HEAD^     回退上一个版本,上上版本为HEAD^^,上100个版本,HEAD~100,往前的版本,直接写commit id
git reflog                 查看命令历史
git checkout -- demo.txt   丢弃工作区内容
git reset HEAD demo.txt    丢弃暂存区内容 
git rm demo.txt            删除版本库中的内容
git remote [-v] 	       查看关联的远程库 -v查看详细
git remote add origin https://github.com/ysy-law/git.git  关联github仓库
git remote rm origin                                      删除origin仓库
git push [-u] origin master                               推送到远程仓库,第一次推送需要加参数
git push origin --delete demo      删除远程分支
git clone https://github.com/ysy-law/git.git              克隆远程仓库,Git支持多种协议,包括https,但通过ssh支持的原生git协议速度最快。
git checkout -b demo       创建demo分支并切换到demo分支上
git checkout -b demo origin/demo 本地创建和远程一样的分支
git pull  抓取分支
git branch --set-upstream demo origin/demo 建立本地分支与远程分支的关联
git branch demo            创建demo分支
git branch [-a]            查看所有分支,添加参数还会加上远程分支
git checkout demo          切换到demo分支
git merge demo             快速模式把demo分支合并到主分支上
git branch -d demo         删除demo分支
git branch -D demo         强行删除分支
git log --graph [--pretty=oneline --abbrev-commit] 分支合并图,加上参数表示精简图
git merge --no-ff -m "说明文字" demo  普通模式合并分支
git stash list             查看保存现场
git stash apply            回复工作现场,stash内容不删除
git stash drop             删除stash内容
git stash                  保存工作现场
git stash pop              回到工作现场
git tag demo [commitid]    打一个标签,添加参数表示给那个版本打标签
git tag -a name -m ''      指定标签信息
git tag -s name -m ''      用PGP签名标签
git tag                    查看所有标签









